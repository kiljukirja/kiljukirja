\documentclass[20pt, a4]{article}
\usepackage[a4paper,verbose]{geometry}


\usepackage[T1]{fontenc}
\usepackage[utf8]{inputenc}
\usepackage[finnish]{babel} % Kieli: Suomi

\author{N.~N}
\title{The booklet}

\begin{document}
\huge{

    \pagebreak

    \begin{center}
    \textbf{\underline{\Huge{Pieni kiljukirja v1.00}}}
    \vspace{180mm}

        Kirjan lähdekoodit saatavilla\\
        \Huge{\textbf{\underline{www.github.com/kiljukirja}}}
    \end{center}

    \pagebreak
    \section{}
    Idea tähän vihkoseen tuli kun aloin kiljua juodessa miettimään kuinka suomessa ei tunnuta olevan tietoisia siitä miten loistava juoma kilju on. Toisin kuin ihmiset saattavat luulla, kilju ei ole laitonta valmistaa, siitä ei lähde näkö eikä siitä oikeintehtynä tule ripulia. Kilju on myös erittäin halpaa, jopa euron litralta.

    Oheiset reseptit ovat 25:lle litralle kiljua. Suosittelen kokeilemaan erilaisia reseptejä itse, tämän vihkosen tarkoituksena on inspiroida lukijaa. 

    Sokerin määrä vaikuttaa suoraan valmiin kiljun vahvuuteen. 

    \begin{center}
    \begin{tabular}{ |c|c|c| } 
     \hline
     Sokerin määrä g<litra & Juoman vahvuus\\
     \hline
     12,5 & 1\% \\ 
     \hline
     120 & 6\% \\ 
     \hline
     300 & 16\% \\ 
     \hline
    \end{tabular}
    \end{center}

    Kiljua suositellaan nautittavaksi mm. iloon, suruun, juhlajuomana, kaveriporukassa tai yksin.

    \pagebreak
    \section{}
    Tarvittavat välineet. 
    Näitä saat mm. Tokmannilta, K-Citymarketeista ja Prismasta.\\
    \begin{itemize}    
        \item Käymisastia 30 litraa
            \begin{itemize}
                \item Tärkein kaikista. Voidaan korvata esim. hanallisella vesikanisterilla, kunhan siihen asentaa kanteen vesilukon tai jättää käydessä korkin hieman raolleen
            \end{itemize}
        \item Vesilukko
            \begin{itemize}
                \item Estää ilman pääsemisen käymisastiaan sisään ja päästää käymisprosessissa syntyvän hiilidoksidin ulos
            \end{itemize}
        \item Lappo
            \begin{itemize}
                \item Letku jonka avulla kiljua saa kätevästi siirreltyä astiasta toiseen
            \end{itemize}
    \end{itemize}

    \pagebreak
    \section{}
    \underline{\textbf{Peruskilju - 13-16\%}}
    \\
    \begin{tabular}{ cc } 

        Vesi & 25L \\
        Sokeri & 6000g\\
        Hiiva&\\
        (Kirkastusaineet)&\\
    \end{tabular}
    \\

    \begin{enumerate}
     \item Steriloi käymisastiasi ja työvälineet käyttämäsi desinfiointiaineen ohjeiden mukaisesti
     \item Kaada käymisastiaasi kiehuvaa vettä ~3 litraa ja sekoittele sokerit siihen
     \item Lisää vettä kunnes käymisastiassa on nestettä yhteensä 25 litraa (ota huomioon käyttämäsi hiivan vaatima lämpötila, lisää sopivan lämpöistä vettä)
     \item Lisää hiiva (toivon mukaan) sen mukana tulleiden ohjeiden mukaisesti

     \item Laita käymisastian kansi kiinni, laita kanteen vesilukko, lisää vesilukkoon vesi ja siirrä käymisastia johonkin tasalämpöiseen paikkaan (vessan nurkka on suosittu paikka)
     \section{}
     \item Odota kunnes vesilukko pulputtaa erittäin harvoin (tähän menee yleensä n. viikko)
     \item (jos käytät kirkastusaineita, niin lisää ne tässä vaiheessa ja toimi niiden mukana tulleiden ohjeiden mukaisesti)
     \item Lapota kilju toiseen astiaan, varo sekoittamasta pohjalle laskeutunutta hiivaa uudestaan kiljun sekaan
     \item Tarkista kirkkaus ja maku, toista lapottaminen tarvittaessa uudestaan
     \item Nauti
    \end{enumerate}
    Kiljun kanssa suositellaan lantringiksi mm. päärynämehua tai mehukatti-mehutiivistettä.


    \pagebreak
    \section{}
    \underline{\textbf{Ollin appelsiinikilju - 15\%}}

    \begin{tabular}{ cc } 
        appelsiini-\\
        täysmehu 100\% & 7 litraa\\
        Sokeri & 6000g\\
        Rusinat & 250g\\
    \end{tabular}
    \\
    \\
    \underline{\textbf{Sekakilju - 15\%}}


    \begin{tabular}{ cc } 
        mansikat & 3000g  \\
        mustikat & 1500g \\
        sokeri & 4600g \\
        rusinat & 800g \\
    \end{tabular}
    \\
    \\
    \underline{\textbf{Horiasima - 14\%}}

    \begin{tabular}{ cc } 
        Sokeri & 3100g \\
        Fariinisokeri & 1500g\\
        Sitruunat & 1000g \\
        Hunaja & 1250g \\
        Rusinat & 800g
    \end{tabular}
    \\

    Erityistä:
    \begin{itemize}
        \item Leikkaa sitruunoista kuoren ja hedelmälihan välissä oleva valkoinen osa pois
    \end{itemize}

    \pagebreak
    \section{}


    \underline{\textbf{inkivääri - 6\%}}

    \begin{tabular}{ cc } 
    Sokeri & 2500g\\
    Inkivääri & 500g\\
    Sitruunat & 5 kpl\\

    \end{tabular}
    \\

    Erityistä:
    \begin{itemize}
        \item Inkiväärit kuoritaan ja pilkotaan pieniksi paloiksi, sitten niitä keitetään parissa litrassa vettä ~10 minuuttia. Tämä mehu laitetaan käymisastiaan. 
        \item On suositeltavaa leikkata sitruunoista kuoren ja hedelmälihan välissä oleva valkoinen osa pois, sillä se voi lisätä kitkerää makua juomaan.
    \end{itemize}
    
    \pagebreak
    \section{}
    \underline{\textbf{Kotikilju - 6\%}}


    \begin{tabular}{ cc } 
        Kotikaljauute & 300g\\
        Sokeri & 2800g\\
        (oluthumalaa)&
    \end{tabular}
    \\
    \\
    Erityistä:
    \begin{itemize}
        \item Seuraa kotikaljauutepakkauksen ohjetta, paitsi että sokeria laitetaan tämän ohjeen mukaan. 
        \item Humalat lisätään käymisastiaan muiden ainesosien mukana.
    \end{itemize}

    \underline{\textbf{Omenakilju - 12\%}}


    \begin{tabular}{ cc } 
        Omenatäysmehu & 16 litraa\\
        Sokeri & 3500g\\
    \end{tabular}
    \\



}


\end{document}